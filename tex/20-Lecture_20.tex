


\section{Lecture 20 - 23.03.21}

\begin{definition}
Let $V$ be an irreducible algebraic variety and $x\in V$. We say $x$ is smooth or regular, or that $V$ is non-singular at $x$ if 
\begin{equation*}
    \dim V = \dim_k T_x V.
\end{equation*}
A point that is not non-singular is called singular. 
\end{definition}

Note: We always have $\dim V \leq \dim_k T_x V$ and $\dim_x V \leq \dim_k T_x V$, where
\begin{equation*}
    \dim_x V = \sup\{\dim V_i\}
\end{equation*}
where $V_i$ is an irreducible component or $V$ containing $x$. 

\begin{example}
Let $V = V(Y^2-X^3)$. We have $\kdim(k[X, Y]/(Y^2-X^3))=1$, and hence that $\dim V = 1$. Lets see how the tangent spaces relate to this dimension. The tangent space at the origin is given by
\begin{align*}
    T_{(0,0)} V 
    &= \Ker J_{(0,0)}(Y^2-X^3) \\
    &= \Ker [-3x^2 (0), 2y(0)] \\
    &= \Ker [0,0] \\
    &= k^2
\end{align*}
while the tangent space at some non zero point $(a, b)\neq (0,0)$ is given by
\begin{align*}
    T_{(a, b)} V
    &= \Ker J_{(a, b)}(Y^2 - X^3) \\
    &= \Ker [-3a^2, 2b] \\
    &\cong k
\end{align*}
We see that $(0,0)$ is the only singular point. 
\end{example}


\begin{theorem}
Let $V\subseteq k^n$ be an irreducible affine algebraic variety. Assume $I(V)=(F_1, \ldots, F_r)$. Then $V$ is non-singular at a point $x$ if and only if $\rank J_x(F_1, \ldots, F_r) = n-\dim V$. 
\end{theorem}
\begin{proof}
We know that $V$ is non-singular at $x$ if $\dim V = \dim_k T_x V$, which is given by 
\begin{align*}
    \dim_k T_x V 
    &= \dim_k\Ker J_x(F_1, \ldots, F_r) \\
    &= n-\rank J_x(F_1, \ldots, F_r)
\end{align*}
where the last equality is due to the rank-nullity theorem from linear algebra. 
\end{proof}

\begin{corollary}
If $F(X, Y)$ is a polynomial without any common factors, i.e. $I(V(F))=(F)$, then a point $(a, b)$ is singular in $V(F)$ if and only if
\begin{equation*}
    \frac{\partial F}{\partial X}(a, b) = 0 = \frac{\partial F}{\partial Y}(a, b)
\end{equation*}
\end{corollary}
\begin{proof}
The proof is an exercise.
\end{proof}
\todo[inline]{Do the proof}

\begin{example}
Let $F(X, Y) = X^3+Y^3-XY$ and $V=V(F)$. Then both
\begin{align*}
    \frac{\partial F}{\partial X} &= 3X^2-Y \\
    \frac{\partial F}{\partial Y} &= 3^2 -X
\end{align*}
are zero at the point $(0,0)$. Thus $V$ is singular at $(0,0)$. 
\end{example}

\begin{proposition}
Let $V\subseteq \P^n(k)$ be an irreducible projective algebraic variety and $x=[x_0:\ldots:x_n]\in V$ with $x_0\neq 0$. Let further $I(V)=(F_1, \ldots, F_r)$ with $F_i$ homogeneous. 

Define
\begin{equation*}
    A(x) = 
    \begin{bmatrix}
    \frac{\partial F_1}{\partial X_0}(x) &\cdots &\frac{\partial F_1}{\partial X_n}(x) \\
    \vdots & &\vdots \\
    \frac{\partial F_r}{\partial X_0}(x) &\cdots &\frac{\partial F_r}{\partial X_n}(x)
    \end{bmatrix}
\end{equation*}
Then $V$ is non-singular at $x$ if and only if $\rank A(x) = n-\dim V$. 
\end{proposition}

\begin{proof}
As $x_0\neq 0$ we can without loss of generality assume that $x = [1:x_1:\cdots :x_n]$. Observe that $x$ is regular if and only if $(x_1, \ldots, x_n)$ is a regular point of $V^b = V\cap D^+(X)$ and $I(V^b) = (F_1^b, \ldots, F_r^b)$. So $(x_1, \ldots, x_n)$ is regular in $V^b$ if and only if
\begin{equation*}
    \rank J_{(x_1, \ldots, x_n)}(F_1^b, \ldots, F_r^b) = n-\dim V
\end{equation*}

Note that if $F$ is a homogeneous polynomial, then
\begin{equation*}
    \frac{\partial F}{\partial X_j}(x) = \frac{\partial F^b}{\partial X_i}(x_1, \ldots, x_n)
\end{equation*}
and so we get that
\begin{equation*}
    A(x) = 
    \begin{bmatrix}
    \frac{\partial F_i}{\partial X_0} &\vert & J_{(x_1, \ldots, x_n)}(F_1^b, \ldots, F_r^b)
    \end{bmatrix}
\end{equation*}
Hence we must have $\rank A(x)\geq \rank J_{(x_1, \ldots, x_n)}(F_1^b, \ldots, F_r^b)$. Then Euler's formula gives us that 
\begin{equation*}
    d\cdot F = \sum_{j=0}^n x_j \frac{\partial F}{\partial X_j}.
\end{equation*}
Now, if $\deg F_i = d_i$, then 
\begin{equation*}
    d_i F_i = \sum_{j=1}^r x_j\frac{\partial F_i}{\partial X_j}(x)
\end{equation*}
which gives us that
\begin{equation*}
    \frac{\partial F_i}{\partial X_0}(x) = d_i F_i(x) - \sum_{j=1}^n x_j \frac{\partial F_i}{\partial X_j}(x)
\end{equation*}
Which again must mean that the first column of $A(x)$ is given by 
\begin{equation*}
    \begin{bmatrix}
    A_1
    \end{bmatrix}
    =
    -\sum x_j
    \begin{bmatrix}
    \frac{\partial F_1}{\partial X_j}(x) \\
    \vdots \\
    \frac{\partial F_r}{\partial X_j}(x)
    \end{bmatrix}
\end{equation*}
which means that $\rank A(x) = \rank J_{(x_1, \ldots, x_n)}(F_1^b, \ldots, F_r^b) = n-\dim V$.
\end{proof}

\begin{problem}
Let $V = V(Y^2 T - X(X-T)(X-\lambda T)\subseteq \P^2(k)$. For what values of $\lambda$ is $V$ smooth?
\end{problem}
\begin{solution}
We look at $T=1$ in affine space. We have 
\begin{equation*}
    J_x(Y^2 T - X(X-1)(X-\lambda) = [-3X^2+2(\lambda-1)X+\lambda, 2Y](x)
\end{equation*}
For $x=(0,0)$ we have 
\begin{align*}
    J_{(0,0)}(Y^2 T - X(X-1)(X-\lambda) 
    &= [-3X^2+2(\lambda-1)X+\lambda, 2Y](0,0) \\
    &= [\lambda, 0]
\end{align*}
So if we set $\lambda = 0$, then $\Ker J_{(0,0)} = k^2$ which means that $V$ is singular at that point. 

For $x=(1,0)$ we have
\begin{align*}
    J_{(0,0)}(Y^2 T - X(X-1)(X-\lambda) 
    &= [-3X^2+2(\lambda-1)X+\lambda, 2Y](1,0) \\
    &= [-3+2(\lambda - 1)+\lambda, 0] \\
    &= ??
\end{align*}
Which means if we set $\lambda = 1$ then $V$ is again singular at $(1,0)$. 

This means that $V$ is smooth, i.e. consists of only regular points when $\lambda \neq 0$ and $\lambda \neq 1$. 
\end{solution}



\subsection{Regular local rings}

Let $(A, m, k)$ be a local ring. Then 
\begin{enumerate}
    \item $m/m^2 (\cong m\otimes_A k)$ is a vector space over $k$.
    \item $\dim_k m/m^2 \geq \kdim A$. 
\end{enumerate}

\begin{definition}
We say $A$ is a regular ring if we have equality in point $2$ above. 
\end{definition}

\begin{example}
The ring $A=k[[X,Y]]$ with $m=(X, Y)$ is a regular ring.
\end{example}

Note that all regular rings are domains.  Also, the number $\dim_k m/m^2$ also equals the minimal number of generators og $m$. Thus $A$ is regular if and only if $m$ is generated by $\kdim A$ number of elements. 

\begin{proposition}
Let $V$ be an algebraic variety and $x\in V$. Then $V$ is regular at $x$ if and only if $\O_{V,x}$ is a regular ring.
\end{proposition}
\begin{proof}
This proof is an exercise.
\end{proof}
\todo[inline]{Do the proof}

\begin{proposition}
A point at the intersection of two irreducible components must be singular.
\end{proposition}
\begin{proof}
This proof is an exercise.
\end{proof}
\todo[inline]{Do the proof}


\subsection{Curves}

Recall that a curve is an equidimensional algebraic variety of dimension $1$. 

\begin{proposition}
Let $C$ be a curve and $x\in C$. Then $x$ is non-singular if and only if $\O_{C,x}$ is a local pricipal ideal domain (PID). 
\end{proposition}

\begin{theorem}
Let $C$ be an irreducible affine curve. Then $C$ is smooth if and only if $\Gamma(C)$ is integrally closed. 
\end{theorem}

\begin{example}
Let $V = V(Y^2-X^3)$. We have that $\frac{Y}{X}\in \Fr(\Gamma(V))$ but $(\frac{Y}{X})^2 -X = 0$, hence not an integral element. This means that $\Gamma(V)$ is not integrally closed, and by the last theorem there must exist some point $x$ such that $V$ is singular at $x$. We already know from an earlier example that this point is $x=(0,0)$.  
\end{example}
