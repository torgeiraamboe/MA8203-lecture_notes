
\section{Exercises}

\subsection{Chapter 1}
\label{ex:chap_1}


\begin{problem}[1.1]
Is the set $\{ (t, \sin t) \vert t\in \R\}$ algebraic?

\begin{solution}
No. If it had been, then we would have a polynomial in one variable that perfectly matches the graph of the sinus function, which means we can describe it by a polynomial. We get this polynomial in one variable by restricting the second variable in $\R[X, Y]$ to a constant. The sinus function has infinitely many roots, and we know one variable polynomials in the real numbers only have a finite set of roots, hence this is impossible. 
\end{solution}
\end{problem}


\begin{problem}[1.2]
Let $V$ be an affine algebraic set and consider $x\notin V$. Show that there is an $F\in k[X_1, \ldots, X_n]$ such that $F(x)=1$ and $F_{\vert V} =0$.

\begin{solution}
Since $V$ is affine algebraic we know that $V=V(I(V))$. Hence $x\notin V$ implies $x\notin V(I(V))$. This means that there exists a polynomial $f\in I(V)\subset k[X_1,\ldots, X_n]$ such that $f(x)\neq 0$. Notice also that since $f\in I(V)$ we have $f_{\vert V}=0$. Set $f(x)=a$. Since $k$ is a field we have an inverse $a^{-1}$ to a. Since $I(V)$ is an ideal we also have $a^{-1}f \in I(V)$. But we also have $a^{-1}f)x=0$, and by setting $F=a^{-1}f$ we are done.
\end{solution}
\end{problem}



\begin{problem}[1.7]
Let $f:k\longrightarrow k^3$ be the map that associates $(t, t^2, t^3)$ to $t$, and let $C$ be the image of $f$. Show that $C$ is an affine algebraic set and calculate $I(C)$. Show $\Gamma (C)\cong k[T]$. 

\begin{solution}
Lets first show that $C$ is affine algebraic. To show this we need at least one polynomial has $C$ as its zero set. By definition we have $C=\Ima (f) = \{ (x, y, z)\in k^3 \vert \exists t\in k \text{ s.t. } x=t, y=t^2, z=t^3\}$, so we want a set of polynomials $\{P(X, Y, Z)\}\subset k[X, Y, Z]$ such that $P(t, t^2, t^3)=0$. We see that one such polynomial is $P=2X^6-Y^3-Z^2$. 

We now calculate $I(C)$. 

Lastly we show that $\Gamma(C)\cong k[T]$. Because $f$ is a map of algebraic varieties, i.e. a regular map, we get a map $f^*$ of finite type reduced algebras, $f^*:\Gamma(k^3)\longrightarrow \Gamma(k)$. We know $\Gamma(k^3)\cong k[X, Y, Z]$ because $I(k^3)=(0)$ as $k$ is assumed infinite. Similarly we have $I(k)=(0)$, hence $\Gamma(k)\cong k[T]$. Hence $f^*$ is a map
\begin{equation*}
    f^*: k[X, Y, Z]\longrightarrow k[T]
\end{equation*}
defined as precomposition. This means that if we take a polynomial $P(X, Y, Z)\in k[X, Y, Z]$, then $f^*(P)=P \circ f$. This means $P(X, Y, Z)\mapsto P(T, T^2, T^3)$, which means that we have $I(C)=\Ker (f^*)$. By the first isomorphism theorem we have $\Ima(f^*)\cong k[X, Y, Z]/\Ker (f^*)$ which means $\Ima(f^*)\cong k[X, Y, Z]/I(C)\cong \Gamma(C)$. We see that $f^*$ is surjective as any one variable polynomial can be described by a three variable polyomial by letting the last two variables be zero. We can also see this as $f$ is injective, with left inverse $(x, y, z)\rightarrow x$. Hence $\Ima(f^*)\cong k[T]$ which means $\gamma(C)\cong k[T]$. 
\end{solution}
\end{problem}


\begin{problem}[1.8]
Assume that $k$ is algebraically closed. Find $I(V)$ when $V$ is given by
\begin{enumerate}
    \item $V_1 = V(XY^3+X^3Y-X^2+Y)$
    \item $V_2 = V(X^2Y, (X-1)(Y+1)^2)$
    \item $V_3 = V(Z-XY, Y^2+XZ-X^2)$
\end{enumerate}

\begin{solution}
\begin{enumerate}
    \item 
    \item Since we need $x^2y = 0$ then we must have $x=0$ or $y=0$. If $x=0$ then by $(x-1)(y+1)^2=0$, we must have $y=-1$, and if $y=0$ we must have $x=1$. Hence $V_2 = V(X, Y+1)\cup V(X-1, Y)$. This means we have $I(V_2)=(X, Y+1)\cap (X-1, Y)$ as they are both prime, and hence radical. 
    \item
\end{enumerate}
\end{solution}
\end{problem}



