
\section{Lecture 17 - 09.03.21}

We continue looking at dimension theory.

\begin{proposition}
Let $X$ be an irreducible algebraic variety and $U\subseteq X$ be a non-empty open subset. Then $\dim X = \dim U$. 
\end{proposition}
\begin{proof}
We first prove the statement for affine algebraic varieties, so let's assume $X$ is affine. Since $U\neq \emptyset$ there exists a distinguished open set $D(f)\subseteq U$, where $f\in \Gamma(X)$. 

We $\dim D(f) \leq \dim U \leq \dim X$ by last lecture, and also that 
\begin{equation*}
    \dim D(f) = \kdim \Gamma(X)_f = \trdg_k \Fr(\Gamma(X)_f)
\end{equation*}
and
\begin{equation*}
    \dim X = \kdim \Gamma(X) = \trdg_k\Fr(\Gamma(X))
\end{equation*}
The only difference between $\Fr(\Gamma(X)_f)$ and $\Fr(\Gamma(X))$ is that $f$ gets inverted earlier, which does not matter in the end, as everything is invertible in the fraction field. Hence $\Fr(\Gamma(X)_f)=\Fr(\Gamma(X))$ which means they have the same transcendence degree. Hence $\dim D(f) = \dim X$, which means $\dim U = \dim X$ when $X$ is affine. 

In the general case, i.e. $X$ not affine, the above proof shows that all non-empty irreducible open affine subsets of $X$ have the same dimension, say $r$. As the distinguished opens form a basis for these affine subsets, we have also that any two sets with non-empty intersection must have the same dimension. 

Assume now that $\dim X> r$. This means that we have a chain $F_0\subsetneq \cdots \subsetneq F_n$ of closed irreducible subsets, where $n>r$. Let $x\in F_0$, contained in some open affine set $U$. Then $U\cap F_0\subsetneq \cdots \subsetneq U\cap F_n$ is a chain of closed irreducible distinct subsets in $U$, meaning that $n = r = \dim U$, and hence that $\dim X = r$. 

Now, if instead of being affine $U$ is any open set, then there exists an open affine set $U'\subseteq U$. Hence $\dim U' \leq \dim U \leq \dim X$, but $\dim U' = \dim X$, hence also $\dim U = \dim X$. 
\end{proof}

\begin{example}[$\P^n(k)$]
By exercise 4 in chapter 1 we know that the union of overlapping irreducible subsets is again irreducible, hence that $\P^n(k)$ is irreducible. Hence we have 
\begin{equation*}
    \dim \P^n(k) = \dim D^+(X_0) = \dim k^n = n
\end{equation*}
\end{example}




\subsection{Dimension and counting equations}

Let $V$ be a $d$-dimensional affine algebraic variety and let $f\in \Gamma(V)$. We want to understand a bit better the dimension of $V(f)$. 

\begin{example}
Let $V=k^3$. Then $\dim V(X) = 2 = 3-1$, $\dim V(X, Y) = 1 = 3-2$ and $\dim V(X, Y, Z) = 0 = 3-3$.
\end{example}

There seems to be a correspondence between the number of variables and the dimension in the above example. If we define the codimension of $V(f)$ in $V$ by
\begin{equation*}
    \codim V(f) = \dim V - \dim V(f)
\end{equation*}
then we can rewrite the above example as 
\begin{itemize}
    \item $\codim V(X) = 1$
    \item $\codim V(X, Y) = 2$
    \item $\codim V(X, Y, Z) = 3$
\end{itemize}
This looks very much like an equality between the number of equation and the codimension of the generated variety. 


\textbf{Question:} When does this correspondence hold true in general? 

Let's try to solve this problem. Our goals are 
\begin{enumerate}
    \item Show that $\codim V(f)$ ``should'' be 1. 
    \item Give some relation between $\codim W$ and the number of equations defining $W$.
\end{enumerate}

\begin{proposition}[Two extremal cases]
\begin{enumerate}
    \item The set $V(f)$ is empty if and only if $f$ is invertible in $\Gamma (V)$. 
    \item The set $V(f)$ contains an irreducible component if and only if $f$ is a zero devisor in $\Gamma(V)$.
\end{enumerate}
\end{proposition}
\begin{proof}
\begin{enumerate}
    \item Assume $V(f) = \emptyset$. Recall that $\Gamma(V) = k[X_1, /ldots, X_n]/I(V) = k[X_1, \ldots, X_n]/\sqrt{I}$. Denote the image of $f$ in this algebra by $F$. We know $V(f)\subseteq V(I)$, hence $\sqrt{I}\subseteq \sqrt{(F)}$. The weak nullstellensatz gives us that if $\sqrt{(F)}\subsetneq k[X_1, \ldots, X_n]$, then $V(f)\neq \emptyset$, hence we must have 
    \begin{equation*}
        \sqrt{(F)}=k[X_1, \ldots, X_n]
    \end{equation*}
    This means that there exists a unit $u$ such that $u^m = F\cdot G$ for some $m$. This is still a unit, which means that $F$ is invertible, which again means that $f$ is invertible. 
    
    Assume now that $f$ is invertible in $\Gamma(V)$. This means that the ideal generated by it is the whole ring, i.e. $(f)=\Gamma(V)$. Hence we have that $I(V(f))=\sqrt{(f)}=\Gamma(V)$, which is only the case for $V(f)=\emptyset$. 
    
    \item Assume $V(f)$ contains an irreducible component. Then we can find a $g$ that vanishes on another component. This gives us that $f\cdot g = 0$, and hence that $f$ is a zero devisor. 
    
    Assume now that $f$ is a zero devisor. Hence there exists a $g\neq 0$ such that $f\cdot g = 0$.Thus $V=V(f)\cup V(g)$, where $V(g)\neq V$ as $g\neq 0$. 
    
    If $V_i$ is an irreducible component of $V$, then
    \begin{equation*}
        V_i = (V(f)\cap V_i)\cup (V(g)\cap V_i).
    \end{equation*}
    Since $V_i$ is irreducible on of these must be empty, meaning that either $V_i\subset V(f)$ or $V(i)\subset V(g)$. But not all irreducible components $V_i$ can be a subset of $V(g)$ as we know that $V(g)\neq V$. Hence there must exist some $i$ such that $V_i\subseteq V(f)$. 
\end{enumerate}
\end{proof}
\todo[inline]{Details on part 2 forward direction}



\begin{definition}
An algebraic variety $X$ is called equidimensional if all of its irreducible components have the same dimension. 
\end{definition}

\begin{theorem}[Geometric Krull's principal ideal theorem]
Let $V$ be an equidimensional affine algebraic variety of dimension $n$. Let further $f\in \Gamma(V)$ be non-invertible and not a zero devisor. Then $V(f)$ is an equidimensional affine algebraic variety with $\codim V(f) = 1$. 
\end{theorem}

\begin{theorem}[Algebraic Krull's principal ideal theorem]
Let $A$ be a commutative noetherian ring and $f\in A$ be non-invertible and not a zero devisor. Then a minimal prime $P$ over $(f)$ has height $1$. 
\end{theorem}

Here being a minimal prime over $(f)$ means that $(f)\subseteq P$ but there are no other prime ideals between them. Having height 1 means that $P$ only contains one other prime ideal. 

\begin{corollary}
If $(A, M)$ is a local ring and $f\in M$ is not a zero devisor, then 
\begin{equation*}
    \kdim A/(f) = \kdim A - 1
\end{equation*}
\end{corollary}

\begin{corollary}
Let $V$ be an equidimensional affine algebraic variety of dimension $n$ and let $f_1, \ldots, f_r \in \Gamma(V)$. If $W$ is an irreducible component of $V(f_1, \ldots, f_r)$, then $\codim W \leq r$. 
\end{corollary}
\begin{proof}
Induct on the codimension $r$. 
\end{proof}



\begin{example}
Let $V=V(X, Y)$ and $f=X(X+Y+1)$. Then $f$ is a zero devisor and $V(f)$ is a non-equidimensional variety consisting of a line and a point. 
\end{example}

\begin{proposition}
Let $X$ and $Y$ be irreducible algebraic varieties in $k^n$ of dimension $r$ and $s$ respectively. Then every irreducible component of $X\cap Y$ has dimension greater than or equal to $r+s-n$.
\end{proposition}

\textbf{Question:} If $W\subseteq V$ has codimension $r$, can we define $W$ by $r$ equations? 

\begin{proposition}
Let $V$ be an irreducible affine algebraic variety such that $\Gamma(V)$ is a unique factorization domain (UFD). Let $W\subseteq V$ be a closed irreducible subset of codimension $1$. Then there exists an element $f\in \Gamma(V)$ such that $W=V(f)$. 
\end{proposition}
\begin{proof}
Since $W$ is irreducible we know that $I(W)$ is a prime ideal in $\Gamma(V)$. Irreducible subsets of $W$ are in one-to-one correspondence with prime ideals in $\Gamma(V)$ containing $I(W)$, which by codimension 1 gives us that the height of $I(W)\leq 1$. 

We claim that $I(W)$ is a principal ideal. 

Let $0\neq g\in I(W)$. Since we are in a UFD we can factorize $g$ into $g=u\cdot f_1\cdot \ldots \cdot f_t$, where $f_i$ are irreducible. Since $I(W)$ is prime then $f_i\in I(W)$. Since we are in a UFD then $(f_i)$ is a prime ideal. Hence we have $(0)\subsetneq (f_i)\subseteq I(W)$. But, this can't happen as we have height $0$ or $1$. Hence there must be some $i$ such that $(f_i)=I(W)$. This then gives us that 
\begin{equation*}
    W = V(I(W))=V(f_i)
\end{equation*}
which is what we wanted. 
\end{proof}

\begin{proposition}
Let $V$ be an irreducible affine agebraic variety, and $W$ an irreducible affine subvariety with $\codim W = r>1$. Then there exists $f_1, \ldots, f_r\in \Gamma(V)$ such that $W$ is an irreducible component of $V(f_1, \ldots, f_r)$.
\end{proposition}

\begin{problem}
The elements $f_1, \ldots, f_r$ in the above proposition are called a system of parameters. Are these related in any meaningful way to the more standard notion of a system of parameters used in algebra? 
\end{problem}