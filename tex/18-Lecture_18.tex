

\section{Lecture 18 - 16.03.21}

\subsection{Morphisms and dimension}

Let $\phi\colon X\longrightarrow Y$ be a morphism of irreducible algebraic varieties. In the more general case of non-irreducible varieties we can always decompose into its irreducible components, so choosing these looses no generality. Recall that a morphism of algebraic varieties is given by 
\begin{align*}
    \phi\colon X &\longrightarrow Y \\
    \phi^\# \colon \O_X &\longrightarrow \phi_*\O_Y
\end{align*}
where $\phi$ is a continuous map of topological spaces and $\phi^\#$ is a morphism of sheaves of rings. 

Let $y\in Y$. We call the set $\phi\inv(y)=\{ x\in X \vert \phi(x)=y \}$ the fiber of $y$. 

Our goal is to compare $\dim X$ and $\dim Y$. These will relate through $\dim \phi\inv(y)$. 

\begin{example}
Let $X=k^{n+d}$, $Y=k^n$ and $\pi\colon X\longrightarrow Y$ be the projection onto the first $n$ coordinates, i.e. $(x_1, \ldots, x_{n+d})\mapsto (x_1,\ldots, x_n)$. Then $\dim X = n+d$, $\dim Y = n$ and
\begin{align*}
    \dim \pi\inv(a_1, \ldots, a_n) 
    &= \dim \{ (a_1, \ldots, a_n, x_{n+1}, \ldots, x_{n+d}) \} \\
    &= \dim k^d \\
    &= d
\end{align*}
This gives us that $\dim X = \dim Y + \dim \pi\inv(a_1, \ldots, a_n)$. 
\end{example}

\textbf{Question:} Does this always work, i.e. do we always have for $y\in Y$ that
\begin{equation}
\label{eq:morphism_dim_rel}
    \dim X = \dim Y + \dim \phi\inv(y)
\end{equation}

\begin{example}[Counterexample]
Consider algebraic varieties $X$ and $Y$ and fix some $y\in Y$. Define $\phi(x)=b$ for all $x\in X$. Then 
\begin{equation*}
\phi\inv(y) = 
\begin{cases}
\emptyset \text{ if } y\neq b \\
X \text{ if } y=b
\end{cases}    
\end{equation*}
Hence we have that $\dim X = \dim\phi\inv(b)$. If we want to fulfill \cref{eq:morphism_dim_rel}, then we must have $\dim Y = 0$, but we can in this case choose $Y$ to be any dimension we want. 
\end{example}

In hindsight the obvious reason that the above counter example does not work while the projection example works is due to $\phi$ not being surjective. So a possible fix is to add a surjectivity criteria. This turns out to actually be a bit too strong of a requirement. 

\begin{example}
Let $V=V(XY-1)\subseteq k^2$ and $\phi\colon V\longrightarrow k$ the projection $\phi(x, y) = x$. What is $\dim \phi\inv (x)?$

If $x=0$ then $\phi\inv(x) = \emptyset$, while if $x\neq 0$ then there exists a unique $y\in k$ that is an inverse to $x$, i.e. $xy = 1$. Hence there is a unique preimage, meaning that $\dim\phi\inv(x) =0$. 

Hence \cref{eq:morphism_dim_rel} holds for all $x\neq 0$. 
\end{example}

Notice that we have $\overline{\Ima\phi} = k$. 

\begin{example}
Let $V=V(XZ-Y)\subseteq k^3$ and $\psi\colon V\longrightarrow k^2$ be the projection $\psi()x, y, z)=(x, y)$. What is $\dim\psi\inv(x, y)$?

If $x=0$ then we must also have $y=0$ to have a non-empty inverse image. But we can let $z$ vary as we want, thus the inverse image of $(0,0)$ is one-dimensional. If $x\neq 0$ and $y = 0$, then we must have $z=0$, meaning that the inverse image of $(x, 0)$ is zero-dimensional. For both $x$ and $y$ non-zero, then we can find a unique $z$ in the preimage, given by $z = x\inv y$.

Hence \cref{eq:morphism_dim_rel} holds for all $x\neq 0$.
\end{example}

Notice that we also here have $\overline{\Ima\psi} = k^2$. This prompts the following definition. 

\begin{definition}[Dominant morphism]
A morphism $\phi\colon X\longrightarrow Y$ between two algebraic varieties $X, Y$ is called dominant if $\overline{\Ima\phi} = Y$. 
\end{definition}

The moral of the above two examples and the definition is that if we have a dominant morphism, then the ``general'' or ``typical'' fiber will have the expected dimension, but fibers over ``special'' points might be different. 

\begin{lemma}
Assume $\phi\colon X\longrightarrow Y$ is a dominant morphism of irreducible algebraic varieties and let $y\in \Ima\phi$ and $Z$ be an irreducible component of $\phi\inv(y)$. Then there are non-empty affine algebraic sets $U\subseteq X$ and $V\subseteq Y$ such that:
\begin{enumerate}
    \item $\phi(U)\subseteq V$
    \item $\phi_{|U}\colon U\longrightarrow V$ is dominant
    \item $y\in V$
    \item $Z\cap U \neq \emptyset$.
\end{enumerate}
\end{lemma}

\begin{proof}
Let $V\subseteq Y$ be an open affine set with $y\in V$. Then $\phi\inv(V)\subseteq X$ is open, and $Z\subseteq \phi\inv(y)\subseteq \phi\inv(V)$. For any point $z\in Z$ there is an open affine set $U\subseteq \phi\inv(V)$ such that $z\in U$, meaning that $Z\cap U \neq \emptyset$.

We claim that $\phi_{|U}$ is dominant. Let $W\subseteq V$ be a non-empty open set (not necessarily affine). If we can show that $W\cap \phi_{|U}(U)\neq \emptyset$ then we have shown that it is dense in $V$, which means it is dominant. As $\phi$ itself is dominant then we know that $W\cap \phi(X) \neq \emptyset$ and hence that $\phi\inv(W)\subseteq X$ is non-empty and open. Since $X$ is irreducible we know that $\phi\inv(W)\cap U\neq\emptyset$, which means that there exists an element $w\in W$ such that $\phi(x)\in W\cap \phi_{|U}(U)$. This means that $W\cap \phi_{|U}(U)\neq \emptyset$ and hence that $\overline{\phi_{|U}(U)} = V$, which is the definition of being dominant. By that we are done. 
\end{proof}

A consequence of this lemma is that we can in most cases reduce to affine algebraic varieties when proving stuff about dimension. This is because $\dim X = \dim U$, $\dim Y = \dim V$ and $\dim Z = \dim(Z\cap U)$, where then $Z\cap U$ is an ireducible component of $\phi_{|U}\inv(y)$. 

Another one is that if $\phi\colon X\longrightarrow Y$ is a dominant morphism, then $\dim Y\leq \dim X$. This is because $\phi$ being dominant gives us that $\phi_*\colon \Gamma(Y)\longrightarrow \Gamma(X)$ is injective, which again implies that 
\begin{equation*}
    \dim Y = \trdg_k \Fr(Gamma(Y)) \leq \trdg_k \Fr(\Gamma(X)) = \dim X
\end{equation*}


\begin{theorem}[The dimension theorem]
Let $\phi\colon X\longrightarrow Y$ be a dominant morphism of irreducible algebraic varieties. 
\begin{enumerate}
    \item Let $y\in Y$. Every irreducible component of $\phi\inv(y)$ has dimension at least $\dim X - \dim Y$.
    \item There exists a non-empty open set $U\subseteq \phi(X)$ such that for all $y\in U$ we have $\dim\phi\inv(y) = \dim X - \dim Y$.
\end{enumerate}
In fact, every irreducible component of $\phi\inv(y)$ is of dimension $\dim X - \dim Y$.
\end{theorem}

\begin{proof}
We only sketch the proof of part 2:

By the previous lemma we can assume that $X$ and $Y$ are affine algebraic varieties. When $\phi$ is dominant we have that $\phi_*\colon \Gamma(Y)\longrightarrow \Gamma(X)$ is injective. The algebra $\Gamma(X)$ is of finite type over $k$ and by $\phi_*$ injective it is also of finite type over $\Gamma(Y)$. Hence
\begin{equation*}
    \Gamma(X) = \Gamma(Y)[b_1, \ldots, b_r, \ldots, b_n]
\end{equation*}
where $b_1, \ldots, b_r$ is a trancendence basis of $\Gamma(X)_{(0)}$ over $\Gamma(Y)_{(0)}$. 

We have 
\begin{align*}
    \Gamma(Y)[b_1, \ldots, b_r] 
    &\cong \Gamma(Y)\otimes_k k[T_1, \ldots, T_r] \\
    &=\Gamma(Y)\otimes_k\Gamma(K^r)
\end{align*}
Which means that $\Gamma(Y)[b_1, \ldots, b_r] = \Gamma(Z)$ for the algebraic variety $Z = Y\times k^r$. Hence we have 
\begin{center}
\begin{tikzcd}
X \arrow[rr, "\phi"] \arrow[rd, "\psi"'] &                                 & Y \\
                                         & Z \arrow[ru, "\pi"', two heads] &  
\end{tikzcd}    
\end{center}
where $\pi$ is the projection onto $Y$. Since $\phi$ is dominant, then $\psi$ has to be as well. 
\begin{enumerate}
    \item First show that there exists a non-empty open set $U\subseteq \phi(X)$. It suffices to show that there exists a non-empty open set $\Omega \subseteq \psi(X)$, as we have that $\pi(\Omega)\subseteq \phi(X)$. 
    \item Using the algebraically dependent elements $b_{r+1},\ldots, b_n$ we get equations 
    \begin{equation*}
        c_{n_i, i}b_i^{n_i}+\cdots + c_0 = 0
    \end{equation*}
    for all $i = r+1, \ldots, n$ where the coefficients lie in $\Gamma(Y)[b_1, \ldots, b_r]$ and the first coefficient is non-zero. Set $0\neq f = \Pi_{i=r+1}^n c_{n_i, i} \in \Gamma(Y)[b_1, \ldots, b_r]$ and let $\Omega = D_Z(f)\subseteq Z$. 
\end{enumerate}

    Then $\psi\inv(D_Z(f)) = D_X(\psi^*(f)\subseteq X$.
    
    The map $\Gamma(Y)[b_1, \ldots, b_r] \longrightarrow \Gamma(X)$ induces an integral extension 
    \begin{equation*}
        \Gamma(Y)[b_1, \ldots, b_r]_f \longrightarrow \Gamma(X)_{\psi^* f}
    \end{equation*}
    hence we get a map
    \begin{equation*}
        \psi: D_X(\psi^* f)\longrightarrow D_Z(f)
    \end{equation*}
    which is surjective by the going up theorem. 
    
    This shows an outline of the proof of existence of almost the set we wanted, but we need to find another set with the correct dimension. Then we intersect these to get the set $U$. 

\end{proof}

