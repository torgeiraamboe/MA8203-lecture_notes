



\section{Lecture 26 - 27.04.21}

\subsection{Schemes}

Let $X$ be a topological space. Recall that a presheaf of abelian groups on $X$ is a functor 
\begin{equation*}
    \F\colon Open(X)^{op}\longrightarrow Ab
\end{equation*}
and that a sheaf is a presheaf such that if an open subset $U\subseteq X$ is covered by open set $\{ U_i\}$, then for any $f_i\in \F(U_i)$ such that $f_{i|U_i\cap U_j} = f_{j|U_i\cap U_j}$ there exists a unique $f\in \F(U)$ such that $f_{|U_i} = f_i$ for all $i$. 

\begin{problem}
Hartshorne uses a slightly different definition of a sheaf. Prove that these are equivalent. 
\end{problem}

\begin{example}
Let $V$ be an affine alebraic set. For $U\subseteq V$, let $\O(U)$ be the ring of regular functions $U\longrightarrow k$. If $U'\subseteq U$, then $\O(U)\longrightarrow \O(U')$ is given by restriction. 

This $\O$ is the sheaf of regular functions on $V$, and is the sheaf we defined for affine algebraic varieties earlier. 
\end{example}

Recall that morphisms of sheaves $\phi\colon \F\longrightarrow \G$ are maps such that the diagram
\begin{center}
\begin{tikzcd}
\F(U) \arrow[r] \arrow[d] & \G(U) \arrow[d] \\
\F(U') \arrow[r]          & \G(U')         
\end{tikzcd}    
\end{center}
commutes for all open set $U'\subseteq U$. 

\begin{proposition}
A morphism of sheaves $\phi\colon \F\longrightarrow \G$ is an isomorphism if and only if the induced map on stalks $\phi_p\colon \F_p\longrightarrow \G_p$ is an isomorphism for all points $p$. 
\end{proposition}

Let $A$ be a commutative ring with identity. Recall that $Spec A = \{p\subseteq A|p \text{ a prime ideal}$. For any ideal $a\subseteq A$ set
\begin{equation*}
    V(a) = \{ p\in Spec A | a\subseteq p \}
\end{equation*}

\begin{lemma}
We have 
\begin{itemize}
    \item $V(ab) = V(a)\cup V(b)$
    \item $V(\sum_{i\in I} a_i) = \cap_{i\in I}V(a_i)$
    \item $V(a)\subseteq V(b)$ if and only if $\sqrt{b}\subseteq \sqrt{a}$. 
\end{itemize}
\end{lemma}
\begin{problem}
Prove the above lemma.
\end{problem}

This defines the Zariski topology on $Spec A$, where the sets $V(a)$ are the closed sets in the topology. 

Now, define a sheaf of rings $\O_{Spec A}$ on $Spec A$ as follows:

For an open set $U\subseteq Spec A$ set 
\begin{equation*}
    \O(U) = \{ \text{functions } s\colon U\to \coprod_{p\in U}A_p | 1. \text{ and } 2.\}
\end{equation*}
where 
\begin{enumerate}
    \item $s(p)\in A_p$
    \item For each $p\in U$ there exists a neighborhood $V\subset U$ around $p$, and elements $a,f\in A$, such that for all $q\in V$ with $f\notin q$ then $s(q)=q/f \in A_q$. 
\end{enumerate}
The last point can be thought of as $s$ being locally a quotient of elements in $A$. 

\begin{example}[$Spec \Z$]

The points correspond to the prime ideals in $\Z$, i.e. the ideal generated by prime numbers. 

We have $(p) = V((p))$ and $V((0)) = Spec\Z$. This means that open sets are sets that miss a finite set of points, for example $U = Spec \Z \setminus (3)$. 

What is then $\O(U) = \O(D((3)))$?
\begin{equation*}
    \{ \frac{a}{3^n}|a\in \Z, n\geq 0 \}=\Z[3^{-1}]\to \O(D((3)))
\end{equation*}
by sending $a/3^n$ to $[s\colon D((3))\to \coprod_{p\in D((3))} \Z_p]$, defined by $s(p) = a/3^n \in \Z_p$. 

\end{example}
\todo[inline]{Visuals of open sets etc}

\begin{proposition}
Let $A$ be a ring and let $\O$ be the sheaf defined above. Then 
\begin{enumerate}
    \item for $p\in Spec A$ we have $\O_p\cong A_p$
    \item for $f\in A$ we have $\O(D(f))\cong A_f$
    \item $\Gamma(Spec A, \O)\cong A$
\end{enumerate}
\end{proposition}

\begin{proof}
Notice that point 2 implies point 3 by letting $f=1$. 

We start by proving 1. For $p\in Spec A$ define 
\begin{align*}
    \phi\colon \O_p &\longrightarrow A_p \\
    s_p&\longrightarrow s(p)
\end{align*}
Injectivity is left as an exercise. For surjectivity we notice that $A_p = \{ a/f | f\notin p \}$. If we take a distinguished open set $D(f)$ which is a neigborhood of $p$, then $a/f\in \O(D(f))$, which means that we can let $s=a/f$. Then $s_p = [D(f), a/f]\mapsto a/f$. 

We now prove 2. Define
\begin{align*}
    A_f &\longrightarrow \O(D(f)) \\
    \frac{a}{f^n} &\longmapsto s\in \O(D(f))
\end{align*}
where $s(p)=a/f^n \in A_p$. This is in fact an isomorphism, but the proof is long and technical. It can be seen in Hartshorne. 
\end{proof}

Note that we could have taken point 2. to be the definition of the sheaf, as it is enough to define it on a basis, which the distinguished open set $D(f)$ do form. 

\begin{definition}[Ringed space]
A ringed space is a pair $(X, \O_X)$, where $X$ is a topological space and $\O_X$ is a sheaf on $X$. 
\end{definition}

\begin{definition}[Morphism of ringed spaces]
A morphism $(f, f^\#)\colon (X, \O_X)\longrightarrow (Y,\O_Y)$ of ringed spaces consists of a continuous map 
\begin{equation*}
    f\colon X\longrightarrow Y
\end{equation*}
and a morphism of sheaves 
\begin{equation*}
    f^\#\colon \O_Y\longrightarrow f_*\O_X
\end{equation*}
\end{definition}

\begin{definition}[Locally ringed space]
A ringed space $(X, \O_X)$ is called a locally ringed space if for all $p\in X$ we have that $\O_{X,p}$ are local rings. 
\end{definition}

\begin{proposition}
If $A$ is a ring, then $(Spec A, \O_{Spec A}$ is a locally ringed space.
\end{proposition}

\begin{proposition}
If $\phi\colon A\longrightarrow B$ is a ring homomorphism, then there is an induced morphism of ringed spaces 
\begin{equation*}
    (\overline{\phi}, \phi^\#)\colon (Spec B, \O_{Spec B})\longrightarrow (Spec A, \O_{Spec A}).
\end{equation*}
If we are given such a map of ringed spaces, the this also induces a morphism of rings. 
\end{proposition}

\begin{definition}[Affine scheme]
An affine scheme is a locally ringed space $(X, \O_X)$ that is isomorphic (as locally ringed spaces) to $(Spec A, \O_{Spec A})$ for some ring $A$. 
\end{definition}

\begin{definition}
A scheme is a locally ringed space $(X, \O_X)$ where every point $x\in X$ has a neighborhood $U$ such that $(U, \O_{X|U}$ is an affine scheme. 
\end{definition}

\begin{problem}
Compare to affine algebraic varieties. 
\end{problem}

