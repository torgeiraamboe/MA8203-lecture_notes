\section{Lecture 2 - 18.01.21}

Last time we intuitively introduced algebraic varieties, looked at several examples and provided a goal for the course - Bézout's theorem. To properly build up the theory from scratch we need to reintroduce what we only covered briefly in the introduction, namely algebraic varieties, or actually to start off, affine algebraic sets. 

\subsection{Affine algebraic sets}

Affine algebraic sets are the building blocks of algebraic geometry. All the other objects are built out of these sets, are inspired by them, or act as generalizations of them. So, to understand algebraic geometry and its important objects like algebraic varieties, projective varieties, planar curves, projective planar curves, affine schemes and schemes, we need to have a firm proper understanding of these building blocks - the affine algebraic sets. 

Let $k$ be a field. We saw the definition of an affine algebraic set last time in the introduction, but for completeness and rigorousness we include it again.

\begin{definition}
Let $S\subset k[X_1, \ldots, X_n]$ be any subset. We define the affine algebraic set generated by $S$ to be $V(S)=\{ x\in k^n \vert P(x)=0,\, \forall P\in S$. 
\end{definition}

\begin{definition}[Affine algebraic set]
We say a subset $X\subset k^n$ is an affine algebraic set if we have $X=V(S)$ for some subset $S\subset k[X_1, \ldots, X_n]$.
\end{definition}

The assignment of the generated affine algebraic set from a subset of polynomials is an order reversing assignment. This means simply that given $S\subset S' \subset k[X_1, \ldots, X_n]$, then we have $V(S')\subset V(S) \subset k^n$.

We saw many geometric examples of such sets last time, so lets try to be a bit more algebraic this time around.  

\begin{example}
Let $1\in k[X_1, \ldots, X_n]$ be the identity polynomial. Then we have $V(\{1\}) = \emptyset$.
\end{example}

\begin{example}
Let $0\in k[X_1, \ldots, X_n]$ be the zero polynomial. Then we have $V(\{0\})=k^n$. 
\end{example}

These two examples seem trivial, but they are will be important soon. They are rarely used, but important to keep in mind. 

\begin{example}
Let $n=1$ and let $S$ consist of a single polynomial $P$. Then $V(S)$ is a finite set. This is because it consists of the zeroes of the polynomial, which is finite because we are working over a field. 
\end{example}

\begin{example}
Let $n=2$, then the affine sets in $k^2$ are
\begin{itemize}
    \item the empty set $\emptyset$
    \item the affine planes
    \item the curves
    \item finite sets of points
\end{itemize}
\end{example}

As $k^n$ itself is an affine algebraic set, we call the sets in the above example the affine algebraic subsets of $k^n$. We can study affine algebraic subsets of any affine algebraic set in general, not just $k^n$. 

The assignment $V(S)$ of some set of polynomials $S\subset k[X_1, \ldots, X_n]$ consists as we know of its common zeroes. So, if $P(x)=0$ for some polynomial $P$ in $S$, then also $aP(x)=0$ for some element $a\in k$. Also, for two polynomials $P$, $Q$ in $S$ then if $P(x)=0=Q(x)$, then also $(P+Q)(x)=0$. So it seems like we can expand a bit this set $S$, and still generate the same affine algebraic set. The below definition and lemma makes this observation rigorous. 

\begin{definition}
Let $S\subset R$ be any subset of a ring $R$. The ideal generated by $S$ is the ideal $(S)=\{ \sum_{i=0}^r a_i f_i \vert a_i \in k[X_1, \ldots, X_n], f_i\in S\}$. 
\end{definition}

\begin{lemma}
The affine algebraic variety generated by a set $S\subset k[X_1, \ldots, X_n]$ is the same as the affine algebraic variety generated by the ideal $(S)$.
\end{lemma}
\begin{proof}
Since the assignment $V(-)$ is order reversing, and we have $S\subset (S)$ then we immediately have $V((S))\subset V(S)$. For the other inclusion we let $x\in V(S)$ be any point. Then by definition we know that $P(x)=0$ for all $P\in S$. Let now $Q\in (S)$. We have 
\begin{align*}
    Q(x) 
    &= \sum_{i=0}^r a_i f_i(x) \\
    &= 0
\end{align*}
since we know that $f_i(x) = 0$ for all $i$. This holds for all elements $Q\in (S)$, hence we have $x\in V((S))$, which proves that $V(S)\subset V((S))$. 
\end{proof}

This fact allows us to reduce the size of the set of polynomials $S$, bue to the following result. 

\begin{proposition}
Every affine algebraic set $X$ is generated by a finite set of polynomials, i.e. $X=V(f_1, \ldots f_r)$.
\end{proposition}
\begin{proof}
Since $k$ is a field we have by the Hilbert basis theorem that $k[X_1, \ldots X_2]$ is a Noetherian ring. In a Noetherian ring all ideals are finitely generated. Since $X$ is an affine algebraic set we know that $X=V(S)$ for some subset $S\subset k[X_1, \ldots, X_n]$. By the previous lemma we know that $V(S)=V((S))$, where now $(S)$ is an ideal. But this we know is finitely generated, i.e. $(S) = (f_1, \ldots, f_r)$ for some $r$, and hence we conclude that $X=V(f_1, \ldots, f_r)$. 
\end{proof}

We remark that we also have $V(f_1, \ldots f_r) = V(f_1)\cap V(f_2)\cap \cdots \cap V(f_r)$. This is relatively easy to convince ourselves of, as the left hand side consists of all points in $k^n$ such that the polynomials vanish simultaneously, and the right hand side consists of the intersection of all points in $k^n$ such that the polynomials vanish individually. We often refer to the affine algebraic set generated by a single polynomial, $V(f)$, as a hypersurface in $k^n$\index{Hypersurface}. 

As ideals in a ring are its nice ``substructures'' and these substructures gets sent by the assignment $V(-)$ to some affine algebraic sets, it would be nice to have a correspondence between substructures of the ring and the substructures of the affine algebraic sets. This turns out to be the case, but we need to know what ``nice substructures'' of affine algebraic sets are. 

\begin{proposition}
The affine algebraic subsets are the closed sets in a topology on $k^n$. 
\end{proposition}

\begin{proof}
For the set of affine algebraic sets to define a topology on $k^n$ we need three things:
\begin{enumerate}
    \item $\emptyset$ and $k^n$ are affine algebraic sets.
    \item arbitrary intersections of affine algebraic sets is again an affine algebraic set.
    \item finite union of affine algebraic sets is again an affine algebraic set. 
\end{enumerate}

We have already seen that $\emptyset$ and $k^n$ are affine algebraic sets, because we have $\emptyset = V(\{ 1\})$ and $k^n=V(\{0\})$, so the first point is done. 

For the second one we will show that $ \bigcap_{i\in I}V(S_i) = V(\bigcup_{i\in I}V(S_i))$. 

For the third point it is enough to show that the union of two affine algebraic sets generated by two ideals is again an affine algebraic set generated by an ideal.

Let $I, J$ be two ideals in $k[X_1, \ldots, X_n]$. We know that the product of the ideals, $IJ$, is contained in both of the ideals, i.e. $IJ\subset I$ and $IJ\subset J$. By the order reversing property of $V(-)$ we have $V(J)\subset V(IJ)$ and $V(I)\subset V(IJ)$. Since both of them are contained, then also their union is, i.e. $V(I)\cup V(J) \subset V(IJ)$. This proves on of the inclusions. For the other one we let $x\in V(IJ)$ such that $x\notin V(I)$. Then by definition there exists a polynomial $P\in I$ such that $P(x)\neq 0$. Since we have $PQ\in IJ$ for all $Q\in J$ we know that $x$ has to be an element of $V(J)$ because $(PQ)(x)=0$, which implies $Q(x)=0$ because $k[X_1, \ldots, X_n]$ is a domain. The same argument shows $x\in V(I)$ if $x\notin V(J)$. Hence we have $V(I)\cup V(J) = V(IJ)$. 
\end{proof}
\todo[inline]{proove second point}

This topology is called the Zariski topology on $k^n$ and is hugely important for this course and for the field of algebraic geometry in general. 

This topology is very different from the standard topology we are used to on $k^n$. To give an example we have that the closed sets in $k^3$ with the Zariski topology are the points, the curves and the planes. These are very ``thin'' and ``small'' sets compared to the closed balls that generate the closed sets in $k^n$ with the standard topology. A nice intuition to have is that the closed sets in the Zariski topology are ``sliced'' out of the space we are in, and that slicing a closed set again yields a closed set. These slices have to be polynomially defined of course. 

\begin{definition}[Standard open sets]
Let $f\in k[X_1, \ldots, X_n]$. The standard open sets of $k^n$ are the sets $D(f)=k^n \setminus V(f)$, i.e. the complements of the hypersurfaces. 
\end{definition}

The important thing about the standard open sets is that they form a basis for the Zariski topology on $k^n$. 

\begin{problem}
Show that the intersection of any two open sets is non-empty. 
\end{problem}
\begin{solution}
Since the standard open sets form a basis for the open sets, it is enough to show that $D(f)\cap D(g) \neq \emptyset$ for some arbitrary polynomials $f, g\in k[X_1, \ldots, X_n]$. This is proven using elementary set theory and the previous properties we have shown. 

\begin{align*}
    D(f)\cap D(g) 
    &= k^n\setminus V(f) \cap k^n\setminus V(g) \\
    &= k^n\setminus (V(f)\cup V(g)) \\
    &= k^n\setminus (V((f))\cup V((g))) \\
    &= k^n\setminus V((f)(g)) \\
    &\neq \emptyset
\end{align*}
where the last equality comes from the fact that $(f)(g)\neq (0)$ as $k[X_1, \ldots, X_n]$ is a domain.  
\end{solution}

\subsection{The ideal of an affine algebraic set}

We are now hopefully starting to sense and feel the duality between the geometry and the algebra. This duality comes through the connection between affine algebraic sets and ideals in rings. Until now we have only passed from the algebra to the geometry by assigning a geometric object to an algebraic one. In order to have a nice duality theory we want some way to also go the other way, i.e. some kind of dual or inverse to $V(-)$. 

\begin{definition}[Ideal of affine subset]
Let $V\subset k^n$ be some subset. The ideal of $V$ is defined to be $I(V)=\{ f\in k[X_1, \ldots, X_n] \vert f(x)=0, \forall x\in V \}$. 
\end{definition}

It is maybe not obvious that this set is an ideal, so lets prove that it is. We define the morphism $r:k[X_1, \ldots, X_n] \longrightarrow \mathcal{F}(V, k)$ by sending a polynomial $P$ to its restriction $P_{\vert V}$. Here $\mathcal{F}(V, k)$ is the ring of ring homomorphisms from $V$ to $k$. We need to show that $r$ is a ring homomorphism. It is because $r(P+Q) = (P+Q)_{\vert V} = P_{\vert V} + Q_{\vert V} = r(P)+r(Q)$ and because $r(PQ) = (PQ)_{\vert V} = P_{\vert V}Q_{\vert V} = r(P)r(Q)$. 

Since the kernel of $r$ is exactly the polynomials in $k[X_1, \ldots, X_n]$ that vanish when restricted to $V$ we have $\Ker(r) = I(V)$. The kernels of ring homomorphisms are ideals, and thus we have shown that $I(V)$ is an ideal. 

\begin{definition}[Coordinate ring]
Let $V\subset k^n$. We define the coordinate ring of $V$ to be the finite type $k$-algebra $\Gamma(V)= \Ima(r)$, i.e. the polynomial functions on $V$. 
\end{definition}

By the first isomorphism theorem we have that $\Gamma(V) \cong k[X_1, \ldots, X_n]/I(V)$. 

Similarily to $V(-)$, the assignment $I(-)$ is order reversing, i.e. if $V\subset V'$ then $I(V')\subset I(V)$. 

\begin{problem}
What is the relationship between $V$ and $V(I(V))$?
\end{problem}
\begin{solution}
By definition we have that $V(I(V))$ is the set of elements $y\in k^n$ such that $P(y) = 0$ for all polynomials $P\in I(V)$. These are exactly the polynomials $P\in k[X_1, \ldots, X_n]$ such that $P(z)=0$ for all $z\in V$. So, if we have $x\in V$ then $P(x)=0$ for all $P\in I(V)$. This also means that $x$ is an element such that $P(x)=0$ for all $P\in I(V)$ which means that $x\in V(I(V))$. Hence we have $V\subset V(I(V))$.  

Now, take a point $x\in V(I(V))$. By definition we have $P(x)=0$ for all polynomials $P\in I(V)$. 
\end{solution}
\todo[inline]{do rest of solution}

\begin{problem}
What is the relationship between $I$ and $I(V(I))$?
\end{problem}
\begin{solution}
Let $I\subset k[X_1, \ldots, X_n]$ be an ideal and $P\in I$. We have $V(I) = \{ x\in k^n \vert f(x)=0 \forall f\in I\}$. In particular we have $P(x)=0$ for all $x\in V(I)$, and thus $P\in I(V(I))$. This shows $I\subset I(V(I))$.

As a counterexample to the equality we can for example take the ideal $I=(X^2-Y)^2\subset k[X, Y]$. Then $V(I)$ looks something like 
\begin{center}
\def\svgwidth{0.4\textwidth}
\input{inkscape/parable.pdf_tex}
\end{center}
We know that the polynomial $P=(X^2-Y)^2$ is zero on this affine algebraic set, but so is $X^2-Y$, which is not in $I$. Hence $I\neq I(V(I))$ in general. 
\end{solution}

\begin{problem}
Show that $I(\emptyset)=k[X_1, \ldots, X_n]$.
\end{problem}
\begin{solution}
This statement is vacuously true. The set $I(\emptyset)$ consists of all polynomials $P\in k[X_1, \ldots, X_n]$ such that $P(x)=0$ for all $x\in \emptyset$, which is true because $\emptyset$ contains no elements. 
\end{solution}

\begin{proposition}
If $k$ is an infinite field then $I(k^n)=(0)$. 
\end{proposition}
\begin{proof}
We will use induction on $n$.

First let $n=1$. Then $I(k)=\{ f\in k[X_1]\vert f(x)=0, \forall x\in k \}$. A non-zero polynomial $f$ has a finite set of roots, hence we can find a point $y\in k$ such that $f(y)\neq 0$ due to $k$ being infinite. Hence if we have $f\in I(k)$ then $f$ has to be zero outside of all of its roots, and hence it must be the zero polynomial. 

Now let $n\geq 1$. Suppose that $f\in k[X_1, \ldots, X_n]\setminus \{0\}$ is non-constant. We can then express $f$ as $f=a_r(X_1, \ldots, X_n)X_n^r +\ldots$ for some $r\geq 1$. By induction we can find $x_1, \ldots, x_{n-1}$ such that $a_r(x_1, \ldots, x_{n-1})\neq 0$. Hence $f(x_1, \ldots, x_{n-1}, X_n)$ has at most $r$ roots. We can then find a point $y$ such that $f(y)\neq 0$ due to $k$ being infinite. 
\end{proof}
\todo[inline]{details}

\begin{problem}
Show that $I(\{(a_1, \ldots, a_n)\}) = (X_1 - a_1, \ldots X_n - a_n)$, i.e. points correspond to maximal ideals. 
\end{problem}

\begin{problem}
Compute $I(-)$ and $V(-)$ for some examples.
\end{problem}


