\section{Lecture 3 - 19.01.21}

To warm up our brains we start the lecture with a problem to solve. 
\begin{problem}
Compute $I(V((X, Y^2)))$ in $k[X, Y, Z]$. 
\end{problem}
\begin{solution}
The algebraic set $V((X, Y^2))$ consists of all points $(x, y, z)\in k^3$ such that $P(x, y, z)=0$ for all polynomials $P\in (X, Y^2)$. Polynomials in $(X, Y^2)$ are of the form $P=aX^n + bY^{2m}$, where $a, b\in k[X, Y, Z]$. For these to be equal to zero at $(x, y, z)$ we need that $x=0=y$. We also have no restrictions on $z$. Hence $V((X, Y^2)) = \{ (0, 0, z)\in k^3 \}$. If we look at $I(V((X, Y^2)))$, then this consists of all polynomials which are zero at all points in $V((X, Y^2)) = \{ (0, 0, z)\in k^3 \}$, i.e. polynomials generated by $X$ and $Y$. Hence we have $I(V((X, Y^2))) = (X, Y)$. 
\end{solution}

\subsection{Irreducibility}

\begin{definition}[Irreducible topological space]
Let $X\neq \emptyset$ be a topological space (for us this means an affine algebraic set with the Zariski topology). We call $X$ irreducible if whenever we can write $X=F\cup G$ with $F, G$ closed subsets of $X$, then either $F=X$ or $G=X$. If this is not the case, then we call $X$ reducible, or decomposable. 
\end{definition}

\begin{theorem}
Let $V\subset k^n$ be an affine algebraic set with the Zariski topology. Then $V$ is irreducible if and only if $I(V)$ is a prime ideal in $k[X_1, \ldots, X_n]$. Equivalently, $V$ is irreducible if and only if $\Gamma(V)$ is a domain. 
\end{theorem}
\label{thm:irreducible_iff_prime}
\begin{proof}
Assume that $V$ is irreducible. We want to show that $I(V)$ is prime, i.e. that if we have two elements $f, g\in k[X_1, \ldots, X_n]$ such that $fg\in I(V)$, then either $f\in I(V)$ or $g\in I(V)$ (or both of course). 

Assume that we have two such elements $f, g$ such that $fg\in I(V)$. Since $(fg) \subset I(V)$ we have by the order reversing property of $V(-)$ that $V(I(V)) \subset V((fg)) = V(fg)$. By the properties of the Zariski topology we know that $V(fg) = V(f)\cup V(g)$, and thus $V\subset V(f)\cup V(g)$. Then $V=(V\cap V(f))\cup (V\cap V(g))$, which are two closed subsets of $V$. Since we assumed that $V$ was irreducible we know that either $V=V\cap V(f)$ or $V=V\cap V(g)$. Assume without loss of generality that $V=V\cap V(f)$. Then we have that $V\subset V(f)$, which by the order reversing property of $I(-)$ means that $I(V(f))\subset I(V)$. And since we know $f\in I(V(f))$ we have finally $f\in I(V)$.

Assume now that $I(V)$ is prime. We will show $V$ irreducible by showing that a given decomposition of $V=V_1\cup V_2$, where $V\neq V_1$, $V\neq V_2$, leads to a contradiction to $I(V)$ being prime. 

Assume $V$ has such a decomposition, i.e. that $V$ is reducible. Since $V_i\subset V$ we have by the order reversing property of $I(-)$ that $I(V)\subset I(V_i)$. Since $V_i\subsetneq V$ we also have $I(V)\subsetneq I(V_i)$, because $V(I(-)) = Id(-)$ implies that $I(-)$ is an injection. Hence there exists $f_1 \in I(V_1) \setminus I(V)$ and $f_2\in I(V_2)\setminus I(V)$. But, notice that $f_1 f_2$ in fact vanishes on $V$ since it vanishes on both $V_1$ and $V_2$ and hence on their union, which is $V$. Thus $f_1 f_2 \in I(V)$, but we explicitly chose $f_1$ and $f_2$ not in $I(V)$, and hence this is a contradiction to $I(V)$ being a prime ideal, meaning that our assumption about $V$ being reducible must have been wrong. 
\end{proof}

\begin{corollary}
If $k$ is an infinite field, then $k^n$ is irreducible. 
\end{corollary}
\begin{proof}
When $k$ is infinite we have previously shown that $I(k^n)=(0)$. Since $(0)$ is a prime ideal in $k[X_1, \ldots, X_n]$, because it is a domain, we have by the previous theorem that $k^n$ must be irreducible. 
\end{proof}

Notice that this is not true in general when $k$ is a finite field.

\begin{theorem}
Let $V\subset k^n$ be a non-empty affine algebraic set. Then there exists a (up to permutation) unique collection of irreducible affine algebraic sets $V_1, \ldots, V_r$ such that $V_i \nsubseteq V_j$ for $i\neq j$, and $V=V_1\cup \cdots \cup V_r$. 
\end{theorem}
\begin{proof}
There are two parts to this proof, showing such a decomposition exists, and showing it is unique. We start by showing existence.

Assume that we have a non-decomposable affine algebraic set $V$. This has a corresponding ideal $I(V)$. Since $k$ is a field, then $k[X_1, \ldots, X_n]$ is Noetherian by the Hilbert basis theorem. This means we can choose $V$ to be the affine algebraic set such that the ideal $I(V)$ is maximal. Here we don't necessarily mean that $I(V)$ is a maximal ideal, but that is the biggest with respect to the property that $I(V)$ is non-decomposable. Since $V$ is assumed non-decomposable we must have a decomposition $V=F\cup G$ of $V$ into closed subsets such that $F\neq V$, $G\neq V$. 

We previously said that $I(-)$ is injective, and hence we have $I(V)\subsetneq I(F)$ and $I(V)\subsetneq I(G)$. Since $I(V)$ was maximal among the ideals of non-decomposable affine algebraic sets we must have that $F$ and $G$ are decomposable. Then we have $F=\bigcup_{i=1}^s V_i$ and $G=\bigcup_{i=s+1}^r V_i$, where $V_i$ is irreducible. This of course gives us a decomposition $V=V_1\cup\cdots\cup V_r$ into irreducible sets. By potentially removing some overlapping sets we also get $V_i\nsubseteq V_j$ for $i\neq j$. Hence all affine algebraic sets are decomposable. 

Lets show that this decomposition is unique up to permutation. 

Assume we have two decompositions $V=V_1\cup\cdots\cup V_r$ and $V=W_1\cup\cdots\cup W_s$. Then we have $V_1 = V\cap V_1 = (W_1\cap V_1)\cup\cdots\cup (W_s\cap V_1)$. Since $V_1$ is irreducible by assumption we must have $V_1=W_j\cap V_1$ for some $j$. This implies $V_1\subseteq W_j$. Similarily we have $W_j = V\cap W_j = (V_1 \cap W_j)\cup\cdots\cup (V_r\cap W_j)$, which by the irreducibility of $W_j$ must mean that $W_j = V_k\cap W_j$ for some $k$, and hence $V_j\subseteq V_k$. 

But this means that we have $V_1\subseteq W_j\subseteq V_k$, which cant be the case as $V_1\nsubseteq V_i$ for all $i\neq 1$. This means that $k=1$, and in turn $V_1=V_k$. Since $W_j$ is squeezed in the middle of two equal sets, it also mus be equal to them, i.e. $V_1=W_j$. We now reorder the decomposition into $V=W_j\cup W_1\cup\cdots W_{j-1}\cup W_{j+1}\cup\cdots\cup W_s$ and repeat the process by using $V_2$ instead. We can continue this process until we have associated every $V_i$ with some $W_j$, which means that $r=s$ and that the two decompositions are the same after some reordering. 
\end{proof}


\subsection{Hilbert's nullstellensatz}

For the rest of this lecture we will assume that $k$ is an algebraically closed field. This is essential to the theorem. We will first prove the weak nullstellensatz and use that to prove the strong one. 

\begin{theorem}[Hilbert's weak nullstellensatz]
Let $I\subsetneq k[X_1, \ldots, X_n]$ be a proper ideal. Then $V(I)\neq \emptyset$. 
\end{theorem}
\label{thm:weak_nullstellensatz}
In class we omitted the proof of this due to proving it in the course MA8202 - Commutative algebra, which this course has as a prerequisite. But, for completeness sake I have added a proof. 
\begin{proof}
Every proper ideal is, or is contained in a maximal ideal. Hence, for some maximal ideal $M\subseteq k[X_1, \ldots, X_n]$ we have $I\subseteq M$. By the order reversing property of $V(-)$ we get $V(M)\subseteq V(I)$, so it is in fact enough to look at maximal ideals. 

In the last lecture we saw that $I\subseteq I(V(I))$ for some ideal $I\subseteq k[X_1, \ldots, X_n]$. If we apply this to $M$ we get $M\subseteq I(V(M))$, which implies either $I(V(M))=M$ or $I(V(M))=k[X_1, \ldots, X_n]$ as $M$ is a maximal ideal. By \cref{thm:irreducible_iff_prime} this means that either $V(M)$ is an irreducible affine algebraic set, as $I(V(M))=M$ is a prime ideal, or that we have $I(V(M))=k[X_1, \ldots, X_n]$ which means $V(M)=\emptyset$. Hence we need to justify that there exists points in $V(M)$. 

Notice that for some point $a=(a_1, \ldots, a_n)\in k^n$ we have that $M_a=(X_1-a_1, \ldots, X_n-a_n)$ is a maximal ideal. If we somehow could prove that all ideals are of this form, then we would be done as $V(M_a)$ would contain $(a_1, \ldots, a_n)$ and hence be non-empty. So lets prove this. 

We define the evaluation morphism as follows: 
\begin{align*}
    e_{a}: k[X_1,\ldots, X_n]&\longrightarrow k \\
    f &\longmapsto f(a) .
\end{align*}
Note that it is a surjective $k$-algebra homomorphism and since $k$ is algebraically closed, it has kernel $M_{a}$. 

Let $M$ be some arbitrary maximal ideal in $k[X_1, \ldots, X_n]$. Then $k[X_1, \ldots, X_n]/M$ is a finitely generated field extension of $k$. By Zariski's lemma, $k[X_1, \ldots, X_n]/M$ is in fact a finite field extension, better known as a finite dimensional vector space. Since $k$ is algebraically closed, there is an isomorphism of $k$-algebras
\begin{equation*}
    k[X_1, \ldots,X_n]/M\longrightarrow k .
\end{equation*}
Now, let $a_i$ denote the image of $X_i$. Then we get that $M_{a}\subseteq M$, which implies $M_{a} = M$ since $M_{a}$ is a maximal ideal. 

Hence we know that any maximal ideal $M$ will have $V(M)\neq \empty$, and hence we are done. 
\end{proof}

Notice here that we used Zariski's lemma in our proof. This states that if $K$ is a finitely generated $k$-algebra, such that $K$ is also a field, then $K$ is a finite field extension of $k$, i.e. a finite dimensional $k$-vector space. The proof is omitted here, but a discussion about the geometry behind it can be found \href{https://aamathematics.wordpress.com/2020/05/13/more-geometric-intuition/}{{\color{blue}on my blog}}. 

Also note that the affine algebraic set $V((X^2+Y^2+1))$ is empty in $\mathbb{R}^2$, even though $(X^2+Y^2+1)$ is prime in $\mathbb{R}[X, Y]$. Hence the weak nullstellensatz does not hold for non-algebraically closed fields. 

\begin{theorem}[Hilbert's nullstellensatz]
Let $I\subseteq k[X_1, \ldots,X_n]$ be an ideal. Then $I(V(I))=\sqrt{I}$. 
\end{theorem}
I didn't quite understand the proof presented in class. I will still put it below, but I have also put a (in my opinion) easier to understand proof in the appendix. It seems to be essentially the same, but some parts are wrapped up nicely in Zariski's lemma. The proof can be found at \cref{A:pf:nullstellenzats}. 
\begin{proof}
Since $k[X_1, \ldots,X_n]$ is noetherian we know that ideals are finitely generated. So, choose generators $I=(P_1, \ldots, P_r)$. We first prove the inclusion $\sqrt{I}\subseteq I(V(I))$.

Let $f\in \sqrt{I}$. By definition this means that $f^m\in I$ for some $m$. Since $f^m\in I$ we know that $f^m$ vanishes on $V(I)$, i.e. $f^m\in I(V(I))$. But, if $f^m$ vanishes on $V(I)$, then so does $f$, hence $f\in I(V(I))$. 

For the other inclusion we let $f\in I(V(I))$, and we want to show that there is a $m$ such that $f^m\in I$. We note that it is enough to show that $Ik[X_1, \ldots,X_n]_f=k[X_1, \ldots,X_n]_f$. This is because we would have $1_{k[X_1, \ldots,X_n]_f}=\sum P_i\frac{Q_i}{f^{n_i}}$, which would imply that $f^m=\sum P_i(Q_if^{m-n_i})$, where $m=max\{n_i\}$. This means that we would have written $f^m$ as a linear combination of parts with a $P_i$ in each summand, which means $f^m\in (\{P_i\})=I$. 

We note that $k[X_1, \ldots,X_n]_f\cong k[X_1, \ldots,X_n, T]/(1-Tf)$. This is like saying that inverting all powers of $f$, which is what localizing at $f$ does, is the same as adding a new variable to the polynomial ring with the property that ``it is the inverse of $f$''. 

Now we have $Ik[X_1, \ldots,X_n]_f = (P_1, \ldots, P_r, 1-Tf)/(1-Tf)$. Set $J=(P_1, \ldots, P_r, 1-Tf)\subseteq k[X_1, \ldots,X_n, T]$. We claim that $V(J)=\emptyset \subset k^{n+1}$. Suppose that this is not the case. This means that there is an element $(x_1, \ldots, x_n, t)\in V(J)$. We have $P_i(x_1, \ldots, x_n)=0$, hence $(x_1, \ldots, x_n)\in V(I)$. This implies that $f(x_1, \ldots, x_n)=0$ as we have chosen $f\in I(V(I))$. But then $(1-Tf)(x_1, \ldots, x_n)\neq 0$ which means that $(x_1, \ldots, x_n, t)\notin V(J)$, which is a contradiction. Hence we must have $V(J)=\emptyset$. 

By the weak nullstellensatz we then have $J=k[X_1, \ldots,X_n, T]$, which means $Ik[X_1, \ldots,X_n]_f=k[X_1, \ldots,X_n]_f$ and by the previous discussion that $f^m\in I$ which by definition means $f\in \sqrt{I}$. 
\end{proof}

One immediate application is that we now have a bijection between the set of affine algebraic sets in $k^n$ and the radical ideals in $k[X_1, \ldots, X_n]$, given by $I(-)$ and $V(-)$. 