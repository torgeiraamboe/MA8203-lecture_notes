
\section{Lecture 12 - 22.02.21}

As we began introducing last time we want to talk about the kernel, cokernel and image sheaves of morphisms of sheaves. We stated that these not necessarily were sheaves, so we needed a way to fix this, which is done by sheafifying the presheaves. 

\subsection{Sheafification}

\begin{proposition}
Let $\F$ be a presheaf on a topological space $X$. There is a sheaf $\F^+$ on $X$, unique up to unique isomorphism, and a map $\Theta\colon\F\longrightarrow\F^+$, such that for any sheaf $\G$ with morphism $\F\longrightarrow\G$ we have a commutative diagram
\begin{center}
\begin{tikzcd}
\F \arrow[d] \arrow[r, "\Theta"] & \F^+ \arrow[ld, "\exists!", dotted] \\
\G                               &                                    
\end{tikzcd}
\end{center}
\end{proposition}
\begin{proof}
For any open set $U\subseteq X$ we define 
\begin{equation*}
    \F^+(U) = \{ s:U\longrightarrow \bigcup_{p\in U}\F_p \,\vert\, \text{1. and 2. holds}\},
\end{equation*}
where 
\begin{enumerate}
    \item $\forall p$ we have $s(p)\in \F_p$
    \item $\forall p$ there exists a neighborhood of $p$, $V\subseteq U$ and $t\in \F(V)$ such that $\forall q\in V$ we have $t_q=s(q)$
\end{enumerate}
Our claim is that $\F^+$ defined this way is a sheaf that satisfies the above proposition. 

For $V\subseteq U$ we have that $\F(U)\longrightarrow \F(V)$ is the actual restruction maps, as we defined $\F^+$ using functions. This means that $\F^+$ is at least a presheaf. 

Assume we have a set with an open cover, i.e. $U=\bigcup_{i} U_i$ and sections $s_i\in \F^+(U_i)$ such that $s_{i\vert U_i\cap U_j} = s_{j\vert U_i\cap U_j}$. We define $s\in \F^+(U)$ to be the map that sends $x\in U$ to $s_i(x)$ when $x\in U_i$. This gives the existence of a glued section. 

Assume that we have two sections $s, s'\in \F^+(U)$ such that $s_{\vert U_i}=s_i = s'_{\vert U_i}$. Since $\{U_i\}$ is a cover we know that all $x\in U$ lie at least in one of the $U_i$'s. We assume $x\in U_i$. Then \begin{equation*}
    s(x)=s_{\vert U_i}(x) = s_i(x)=s'_{\vert U_i}(x) = s'(x)
\end{equation*}
holds for all points $x\in U$, hence $s=s'$. This shows that the map $s$ we defined above is unique, and hence that $\F^+$ has unique gluing, and is thus a sheaf.  

The map $\Theta$ is defined as follows. For every open set $U\subseteq X$ define $\Theta(U)$ as
\begin{align*}
    \F(U)&\longrightarrow \F^+(U) \\
    s&\longmapsto [U\rightarrow \cup \F_p, x\mapsto s_x]
\end{align*}
\end{proof}
\todo[inline]{Check that sheaves factor through theta}

As mentioned last time we define $\F^+$ to be the sheafification of a presheaf $\F$.


\begin{problem}
Prove that for all $p\in X$ that $\F^+_p = \F_p$.
\end{problem}
This means that at the level of stalks, working with the sheafification is relatively easy.

Let $\phi_\F\longrightarrow \G$ be a morphism of sheaves (of abelian groups). Recall that $\Ker\phi$, $\Ima_{pre}\phi$ and $\Cok_{pre}\phi$ are presheaves. The kernel sheaf $\Ker\phi$ is in fact also a sheaf. 

\begin{definition}
A subsheaf of a sheaf $\F$ on $X$ is a sheaf $\F'$ such that for all open sets $U\subseteq X$ we have that $\F'(U)$ is a subobject of $\F(U)$, and that the restriction maps in $\F'$ are induced by the ones in $\F$. 
\end{definition}
Note that this makes $\F'_p$ a subobject of $\F_p$. 

\begin{definition}
We say a morphism $\phi\colon\F\longrightarrow\G$ is injective if $\Ker\phi=0$. 
\end{definition}
Note that $\phi$ is injective if and only if $\phi(U)$ is injective for all open sets $U\subseteq X$. 

\begin{definition}
We define the image sheaf of a morphism $\phi\colon\F\rightarrow\G$ to be $\Ima\phi = \Ima_{pre}\phi^+$, i.e. the sheafification of the image presheaf. 
\end{definition}
Calling it the image is justified as we have an injective morphism $\Ima\phi\longrightarrow\G$. This map exists because of universal property of the sheafification
\begin{center}
\begin{tikzcd}
\F \arrow[rr, "\phi"] \arrow[rd] &                                              & \G \\
                                 & \Ima_{pre}\phi \arrow[ru, "\psi'"] \arrow[d] &    \\
                                 & \Ima\phi \arrow[ruu, "\exists\psi"', dotted] &   
\end{tikzcd}
\end{center}
It is injective because $\psi'(U)$ is injective for all $U$, hence $\psi'$ injective. This implies $\psi'_p\colon(\Ima_{pre}\phi)_p\longrightarrow \G_p$ is injective, which by the problem above, i.e. $(\Ima_{pre}\phi)p = \Ima\phi_p$, means that $\psi_p$ is injective as well. Being injective on all germs is sufficien to be injective as morphism of sheaves, hence $\psi$ is injective. 

\begin{definition}
We say a morphism $\phi\colon\F\longrightarrow\G$ is surjective if $\Ima\phi = \G$. 
\end{definition}

\begin{definition}
Let $\F'$ be a subsheaf of a sheaf $\F$ on a topological space $X$. We define the quotient sheaf $\F/\F'$ by sending an open set $U\subseteq X$ to $\F(U)/\F'(U)$. 
\end{definition}
\begin{problem}
Show that this is in fact a sheaf.
\end{problem}

\begin{definition}
Let $\phi\colon\F\longrightarrow\G$ be a morphism of sheaves. We define its cokernel sheaf to be the sheaf $\Cok\phi = \Cok_{pre}\phi^+$, i.e. the sheafification of the cokernel presheaf. 
\end{definition}

\begin{definition}
A sequence of sheaves on a topological space $X$,
\begin{equation*}
    \cdots \overset{\phi^{i-2}}\longrightarrow \F^{i-1}\overset{\phi^{i-1}}\longrightarrow \F^i\overset{\phi^i}\longrightarrow \F^{i+1}\overset{\phi^{i+1}}\longrightarrow\cdots
\end{equation*}
is called exact at degree $i$ if $\Ker\phi^{i} = \Im\phi^{i-1}$. The sequence is called exact if it is exact at all $i$. 
\end{definition}

Note that the sequence 
\begin{equation*}
    \cdots \overset{\phi^{i-2}}\longrightarrow \F^{i-1}\overset{\phi^{i-1}}\longrightarrow \F^i\overset{\phi^i}\longrightarrow \F^{i+1}\overset{\phi^{i+1}}\longrightarrow\cdots
\end{equation*}
is exact if and only if the sequence 
\begin{equation*}
    \cdots \overset{\phi^{i-2}(U)}\longrightarrow \F^{i-1}(U)\overset{\phi^{i-1}(U)}\longrightarrow \F^i(U)\overset{\phi^i(U)}\longrightarrow \F^{i+1}(U)\overset{\phi^{i+1}(U)}\longrightarrow\cdots
\end{equation*}
is exact for all open sets $U\subseteq X$.

\begin{definition}
Let $\F$ and $\G$ be sheaves on a topological space $X$. For an open set $U\subseteq X$ we define $\Hom(\F, \G)(U) = Mor(\F_{\vert U}, \G_{\vert U})$. 
\end{definition}

\begin{proposition}
The assignment $\Hom(\F, \G)$ defines a sheaf on $X$. 
\begin{proof}
We first see that it is a presheaf. Take a subset $V\subseteq U$, we need to define a map $Mor(\F_{\vert U}, \G_{\vert U})\longrightarrow Mor(\F_{\vert V}, \G_{\vert V})$. For any subset $W\subseteq V$ we have that $\F_{\vert V}(W)=\F_{\vert U}(W)$, as the extra restriction of the sheaf itself does not do anything since we are looking at function on an even smaller set. Hence we get a map $\F_{\vert V}(W)\longrightarrow \G_{\vert V}(W)$ from the map we already have from $\F_{\vert U}(W)\longrightarrow \G_{\vert U}(W)$. 

Let now $U=\bigcup U_i$ be an open cover and suppose $s_i\in \Hom(\F, \G)(U_i)$ such that $s_{i\vert U_i\cap U_j} = s_{j\vert U_i\cap U_j}$. For $V\subseteq U$ set $V_i=V\cap U_i$, which means that $V=\bigcup V_i$ is an open cover. 

For $f\in \F(V)$ we can restrict it to $V_i$ to get $f_{\vert V_i}\in \F(V_i)$. We can map this to $\G(V_i)$ by using $s_i(V_i)$ to get some $g_i\in \G(V_i)$. Since $\G$ is a sheaf we can glue these $g_i$ to get an unique section $g\in \G(V)$. We then simply define $s\in \Hom(\F, \G)(U)$ by $s(V)(f)=g$.

This defines $s\colon \F_{\vert U}\longrightarrow \G_{\vert U}$ such that $s_{\vert U_i}=s_i$, and thus we have existence of a glued section.

For uniqueness we assume there exists $s, t\in \Hom(\F, \G)(U)$ such that $s_{\vert U_i}=t{\vert U_i}$. For $V\subseteq U$ we have a diagram

\begin{center}
\begin{tikzcd}
\F_{\vert U}(V) \arrow[d] \arrow[r, "t(V)"', bend right] \arrow[r, "s(V)", bend left] & \G_{\vert U}(V) \arrow[d] \\
\F(V_i) \arrow[r, "s(V_i)=t(V_i)"']                                                   & \F(V_i)                  
\end{tikzcd}    
\end{center}
where the vertical arrows are restriction maps. Let $f\in \F(V)$. We want to compare $s(V)(f)$ and $t(V)(f)$. For all $i$ we have
\begin{equation*}
    s(V)(f)_{\vert V_i} = s(V_i)(f_{\vert V_i}) = t(v_i)(f_{\vert V_i})=t(V)(f)_{\vert V_i}
\end{equation*}
which by $\G$ being a sheaf means that $s(V)(f)=t(V)(f)$. So $s(V)$ and $t(V)$ are pointwise equal, i.e. the same map, hence $s(V)=t(V)$. This holds for all open sets $V$, hence also $s$ and $t$ are pointwise equal, making them again equal, i.e. $s=t$. Hence the gluing is unique and we are done.
\end{proof}
\end{proposition}

\begin{definition}
Let now $\F$ and $\G$ be two $\O_X$ modules. We can define their tensor product $\F\otimes_{\O_X}\G$ to be the sheaf associted to the presheaf $U\longrightarrow \F(U)\otimes_{\O_X(U)\G(U)}$. 
\end{definition}



\subsection{Sheaves and varieties}

Let $V$ be an affine algebraic set. We want to study some special ringed spaces, $(V, \O_V)$, which will be the affine algebraic varieties. In more generality we will have algebraic varieties which will be ringed spaces $(X, \O_X)$ that are locally affine. If $V\subseteq \P^n(k)$ is projective we will get projective algebraic varieties, which will be examples of these more general algebraic varieties. 

In even more generality, these will all be examples of schemes, which also are locally ringed spaces. 