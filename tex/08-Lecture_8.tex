
\section{Lecture 8 - 08.02.21}

\subsection{Projective algebraic sets}
A problem we run into while trying to define projective algebraic sets if we were to do it similarly to affine algebraic sets is that polynomials are not always functions on $\P^n(k)$. 

\begin{definition}[Projective zero of a polynomial]
Let $\x\in \P^n(k)$ and $F\in k[X_0, \ldots, X_n]$. We say $\x$ is a zero of $F$, sometimes called a projective zero, if $F(\lambda x)=0$ for all $\lambda \in k^\times$. We write $F(\x)=0$ even though $F$ is not necessarily a function.  
\end{definition}

\begin{proposition}
If $F\in k[X_0, \ldots, X_n]$ is a homogeneous polynomial and $F(x)=0$ for some $x$, then $F(\x)=0$, i.e. $\x$ is a zero of $F$. 
\label{lec8:prop1}
\end{proposition}

\begin{proposition}
Decompose the polynomial $F$ into its homogeneous components, i.e. $F=F_0+F_1+\ldots + F_r$, where $r=deg(F)$ and $F_i$ is the degree $i$ homogeneous component. Then $F(\x)=0 \iff F_i(\x)=0$ for all $i$. 
\label{lec8:prop2}
\end{proposition}

\begin{problem}
Proove this.
\end{problem}

\begin{definition}[Projective algebraic set]
Let $S\subseteq k[X_0, \ldots, X_n]$ be a subset. We call 
\begin{equation*}
    \Vp(S) = \{ \x \in \P^n(k) \vert F(\x)=0, \forall F\in S \}
\end{equation*}
the projective algebraic set of $S$. 
\end{definition}

We note that $\Vp(S) = \Vp((S))$, where $(S)$ if the ideal generated by $S$. This is exactly the same as for the affine case. As $k[X_0, \ldots, X_n]$ is noetherian, we can by Hilbert's basis theorem assume that $S$ is a finite set. 

\begin{example}
$\Vp((0)) = \P^n(k)$
\end{example}

Before we see the next example we need one more definition. 

\begin{definition}[The irrelevant ideal]
The ideal $R^+ = (X_0, \ldots, X_n) \subset k[X_0, \ldots, X_n]$, i.e. the ideal generated by the indeterminates, is called the irrelevant ideal. 
\end{definition}

\begin{example}
$\Vp(R^+) = \emptyset$. This is because $0\notin \P^n(k)$. 
\end{example}

\begin{example}
Let $\x = [x_0:\cdots:x_n] \in \P^n(k)$. As $0\notin \P^n(k)$ we can without loss of generality assume that $x_0 \neq 0$. This means that $\x=[1:x_1/x_0:\cdots:x_n/x_0]$. We can relabel to get $\x=[1:x_1:\cdots:x_n]\in \P^n(k)$. 

Then $\{\x\} = \Vp(X_1-x_1X_0, \ldots, X_n-x_nX_0)$. 
\end{example}

In the affine case we had $\{x\}= V(X_0-x_0, \ldots, X_n-x_n)$, but when we want projective algebraic sets we need homogeneuity. This we can get y multiplying by $X_0$, which acts as $1$. 

As in the affine case we have some standard properties these projective algebraic sets satisfy:
\begin{enumerate}
    \item If $S\subseteq S'$ then $\Vp(S')\subseteq \Vp(S)$, i.e. $\Vp(-)$ has the order reversing property.
    \item $\cap_{i\in I}\Vp(S_i) = \Vp(\cup_{i\in I}S_i)$ for an arbitrary index set $I$.
    \item $\cup_{i=1}^n \Vp(S_i) = \Vp(\prod_{i=1}^n)S_i$.
\end{enumerate}

Together with the previous remark that $\P^n(k)$ and $\emptyset$ lie in the image of $\Vp(-)$ we still have the same Zariski topology on $\P^n(k)$, where the projective algebraic sets are the closed sets. 

\subsection{Ideal of a projective algebraic set}

\begin{definition}[Ideal of a projective set]
Let $V\subset \P^n(k)$ be a subset. We call 
\begin{equation*}
    \Ip(V) = \{ F\in k[X_0, \ldots, X_n] \vert F(\x)=0, \forall \x\in V \}
\end{equation*}
the ideal of $V$. 
\end{definition}

\begin{definition}[Homogeneous ideal]
We call an ideal a homogeneous ideal if it is generated by homogeneous elements. 
\end{definition}

\begin{enumerate}
    \item If $V\subset V'$ then $\Ip(V')\subseteq \Ip(V)$. 
    \item $\Ip(V)$ is a homogeneous and radical ideal.
    \item If $V$ is a projective algebraic set, then $\Vp(\Ip(V))=V$.
    \item If $I$ is an ideal, then $I\subseteq \Ip(\Vp(I))$.
    \item $\Ip(\P^n(k))=(0)$. 
    \item $\Ip(\emptyset) = k[X_0, \ldots, X_n]$.
    \item Irreducibly makes sense. 
\end{enumerate}

\subsection{The cone construction}

The cone construction is a technique for reducing the case of projective algebraic sets to affine algebraic sets. If we let $V\subseteq \P^n(k)$ be a projective algebraic set, and $\pi$ the canonical projection
\begin{equation*}
    \pi\colon k^{n+1}\setminus \{0\} \longrightarrow \P^n(k)
\end{equation*}
then the cone of $V$ is defined to be the set $\Cone(V)=\pi^{-1}(V)\cup\{0\}$. 

The cone of a projective algebraic set has the following two properties. \begin{enumerate}
    \item If $I\subsetneq k[X_0, \ldots, X_n]$ is a proper homogeneous ideal, then $\Cone(\Vp(I)) = V(I)\subseteq k^{n+1}$. 
    \begin{proof}
    
    \end{proof}
    \todo[inline]{proof}
    
    \item If $I=k[X_0, \ldots, X_n]$, then $\Cone(\Vp(I))=\Cone(\emptyset)=\{0\}$. 
    \begin{proof}
    We know that $\Vp(k[X_0, \ldots, X_n]) = \emptyset$, which means that
    \begin{equation*}
        \Cone(\Vp(k[X_0, \ldots, X_n]))=\Cone(\emptyset)=\pi^{\emptyset}\cup \{0\}.
    \end{equation*}
    The inverse image of the empty set under the canonical projection is again the empty set, thus we have $\Cone(\Vp(k[X_0, \ldots, X_n]))=\{0\}$. 
    \end{proof}
\end{enumerate}

\begin{theorem}[Projective nullstellensatz]
Let $k$ be an algebraically closed field and $I\subseteq k[X_0, \ldots, X_n]$ a homogeneous ideal. Then $\Vp(I)=\emptyset$ if and only if $R^+\subseteq \sqrt{I}$. If $\Vp(I)\neq \emptyset$, then $\Ip(\Vp(I))=\sqrt{I}$. 
\end{theorem}
\begin{proof}
Notice that if $I=k[X_0, \ldots, X_n]$, then $R^+\subseteq\sqrt{k[X_0, \ldots, X_n]}=k[X_0, \ldots, X_n]$, i.e. the first statement holds. Hence we can assume that $I$ is a proper ideal. Also notice that $\Vp(I)=\emptyset$ if and only if $\Cone(\Vp(I))=\{0\}=V(I)$.

Let's prove the first statement. Assume that $\Vp(I)=\emptyset$. Then 
\begin{equation*}
    R^+=(X_0, \ldots, X_n)\subseteq I(0)=I(V(I)),
\end{equation*}
where the last equality is by the above remark. By the nullstellensatz for affine algebraic sets we know that $I(V(I))=\sqrt{I}$, hence $R^+\subseteq \sqrt{I}$. 

Assume now that $R^+\subseteq \sqrt{I}$. By the affine nullstellensatz we have $\sqrt{I}=I(V(I))$, hence we have $R^+\subseteq I(V(I))$. Then by the order reversing property of $V(-)$ we have
\begin{equation*}
    \emptyset \neq V(I) = V(I(V(I))) \subseteq V(R^+)=\{0\}.
\end{equation*}
This means that $V(I)=\{0\}$ which means that $\Vp(I)=\emptyset$ by the above remark.

We now prove the second part, i.e. that if the projective algebraic set of an ideal is non-empty, then $\Ip(\Vp(I))=\sqrt{I}$. Assume $\Vp(I)\neq \emptyset$. This gives us that $\Ip(\Vp(I))\subseteq k[X_0, \ldots, X_n]$ is a proper ideal. We claim that $\Ip(\Vp(I))=I(\Cone(\Vp(I)))$. We show both containments. 

Let $F\in \Ip(\Vp(I))$. This means that $F(\x)=0$ for all $\x\in\Vp(I)$, i.e. that $F(\lambda x)=0$ for all $\lambda\in k^\times$. By letting $\lambda=1$ we see that $F(x)=0$, which means that $F\in I(\pi^{-1}(\Vp(I)))$. As $\Ip(\Vp(I))$ is a proper homogeneous ideal we also see that $F(0)=0$, as $F$ can't be a constant, i.e. a homogeneous polynomial of degree 0. This means that $F\in I(\Cone(\Vp(I)))$. Hence we have $\Ip(\Vp(I))\subseteq I(\Cone(\Vp(I)))$. 

Let $F\in I(\Cone(\Vp(I))$. This means that $F(x)=0$ for all $x\in \Cone(\Vp(I)) = V(I)$, as $I$ is a proper homogeneous ideal. We can decompose $F$ into its homogeneous components, i.e. $F=F_0+\cdots+F_r$, where $r=deg(F)$. As $F(0)=0$ we know that $F_0=0$. Thus $F(\lambda x)=\lambda F_1(x)+\cdots \lambda^r F_r(x)$. All these $F_i(x)$ must vanish since $F(x)=0$, hence $F_i(x)=0$. By \ref{lec8:prop2} this means that $F(\x)=0$ and thus $F\in \Ip(\Vp(I))$. Hence $I(\Cone(\Vp(I)))\subseteq \Ip(\Vp(I))$.

We then finally have
\begin{equation*}
    \Ip(\Vp(I))=I(\Cone(\Vp(I))) = I(V(I))=\sqrt{I},
\end{equation*}
where the last equality is by the affine nullstellensatz and the middle equality is due to the properties we described above. 
\end{proof}

\begin{proposition}
There is a bijection between non-empty projective algebraic sets $V\subseteq \P^n(k)$ and homogeneous radical ideals in $k[X_0, \ldots, X_n]$ not containing $R^+$.  
\end{proposition}

As in the affine case we have that irreducible projective algebraic sets correspond to the prime ideals. Also as in the affine case we can compare to certain $k$-algebras. If $I$ is a homogeneous radical ideal not containing $R^+$ corresponding to an affine algebraic set $V$, then $k[X_0, \ldots, X_n]/I$ is a $k$-algebra, which we denote by $\Gamma_{homog}(V)$. These are often called the homogeneous coordinate ring of $V$. For the affine case we had that points on an affine algebraic set corresponded to maximal ideals in $\Gamma(V)$, but for the projective case this is not the case. 

\begin{definition}
Let $V\subseteq \P^n(k)$ be a projective algebraic set and $f\in \Gamma_{homog}(V)$ be homogeneous with positive degree. We define $D^+(f) = \{ \x\in V\vert f(\x)\neq 0 \}$.  
\end{definition}

These are open sets in the Zariski topology on $V$. 

\begin{example}
$\P^n(k) = D^+(X_0)\cup D^+(X_1)\cup \cdots D^+(X_n)$.
\end{example}

Each one of these $D^+(X_i)$ are isomorphic to $k$, which means that projective space is locally affine! These are far from being disjoint unions, but they do form a covering. 

\begin{example}
$\P^2(\R) = D^+(X)\cup D^+(Y) \cup D^+(Z)$. Here $D^+(X)$ consists of all points $[x:y:z]\in \P^2(\R)$ such that $x\neq 0$, and the other ones are the same for the other coordinates. As these are projective coordinates we can look at $[1:y/x:z/x]$ instead, which means that $D^+(X)$ is isomorphic to $\R^2$ through $\phi([x:y:z]) = (y/x, z/x)$, and $\phi^{-1}(a,b)=[1:a:b]$. 
\end{example}

\begin{problem}
Lets take the affine algebraic set $V(Y-X^3)\subseteq \R^2$. We can make it projective by instead studying $\Vp(X^3-YZ^2)\subseteq \P^2(\R)$. Call this projective algebraic set $V$. What does $V$ look like?

\begin{solution}
We try to understand how it looks like on each of the open sets $D^+(X)$, $D^+(Y)$ and $D^+(Z)$. On $D^+(X)$ it looks like $V(YZ^2-1)$, on $D^+(Y)$ like $V(Z^2-X^3)$ and on $D^+(Z)$ like $V(Y-X^3)$. 
We can visualize them as:

We see that all three of them contain the point $[1:1:1]$. 
\end{solution}
\end{problem}
\todo[inline]{Visualization}

\begin{problem}
Find some other points and try to understand how these three open sets glue together globally. 
\end{problem}

